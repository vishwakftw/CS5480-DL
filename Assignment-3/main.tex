\documentclass{article}
\usepackage[utf8]{inputenc}
\usepackage[margin=0.8in]{geometry}
\usepackage{amsmath}
\usepackage{amssymb}
\usepackage{graphicx}
\usepackage{float}
\usepackage{hyperref}
\newcommand{\Do}{\partial}
\newcommand{\G}{\nabla}

\graphicspath{{images/}}

\title{Solutions to the Assignment - 3 : CS5480 - \\
Deep Learning}
\author{Vishwak Srinivasan\\
\texttt{CS15BTECH11043}}
\date{}

\begin{document}
\maketitle

\section*{Question 1}
\begin{flushleft}
The sample text used for this purpose is a 1565 character lyric of the song ``Speed of Sound'' by Coldplay. The lyrics were obtained from the website \href{https://songmeanings.com}{SongMeanings}.

There were 42 distinct characters.
\end{flushleft}

\section*{Question 2}
\subsection*{Part a}
\begin{flushleft}
Denote the hidden state at time \(t\) by \(h_{t}\), and the output by \(o_{t}\) and the input as \(x_{t}\). The notation for the weights will follow from the question. The question expects no biases, hence the equations will not contain them.
\begin{itemize}
\item [\textbf{i.}]
\begin{gather}
\text{Hidden Layer with activation: }h_{t} = \tanh(W_{xh}x_{t} + W_{hh}h_{t-1}) \\
\text{Output Layer without activation: }o_{t} = W_{ho}h_{t} \\
\text{Output Layer with activation: }\hat{y}_{t} = \mathrm{softmax}(o_{t})
\end{gather}

\item [\textbf{ii.}]
Denote the outer-product operator as \(\otimes\) and the Hadamard product as \(\odot\).

Since we are considering a cross-entropy loss (denoted by \(L\)), it is easy to see that the gradient with respect to the logits i.e., \(\hat{o}_{t}\), can be given by:
\begin{equation}
\left(\G_{o^{t}} L\right)_{i} = \frac{\Do L}{\Do L^{t}} \frac{\Do L^{t}}{\Do o_{i}^{t}} = \hat{y}_{i}^{t} - \mathbf{1}_{i = y^{t}}
\end{equation}
where \(\mathbf{1}_{p}\) denotes \(1\) if predicate \(p\) is satisfied and \(0\) otherwise.

Similarly, backpropping to the hidden state \(h_{t}\) gives us (using matrix calculus):
\begin{equation}
\label{hidden}
\G_{h^{t - 1}} L = \left(\frac{\Do h^{t}}{\Do h^{t - 1}}\right)^{T} (\G_{h^{t}} L) + \left(\frac{\Do o^{t - 1}}{\Do h^{t - 1}}\right)^{T} (\G_{o^{t - 1}} L) = W^{T} (\G_{h^{t}} L) \odot (1 - (h^{t})^{2}) + W_{ho}^{T} (\G_{o^{t - 1}} L)
\end{equation}

Now, say we go back to timestep \(k\) from timestep \(T\) backward. Due to this, by standard backprop through time, we get:
\begin{gather}
\displaystyle \G_{W_{ho}} L = \sum_{t=T}^{t=k} \sum_{i} \frac{\Do L}{\Do o_{i}^{t}} \G_{W_{ho}} o_{i}^{t} = \sum_{t=T}^{t=k} (\G_{o^{t}} L) (h^{t})^{T} = \sum_{t=T}^{t=k} (\G_{o^{t}} L) \otimes h^{t} \\
\label{hidden-full}
\displaystyle \G_{W_{hh}} L = \sum_{t=T}^{t=k} \sum_{i} \frac{\Do L}{\Do h_{i}^{t}} \G_{W_{hh}} h_{i}^{t} = (1 - (h^{t})^{2}) \odot \left((\G_{h^{t}} L) \otimes (h^{t-1})^{T}\right) \\
\displaystyle \G_{W_{xh}} L = \sum_{t=T}^{t=k} \sum_{i} \frac{\Do L}{\Do h_{i}^{t}} \G_{W_{xh}} h_{i}^{t} = (1 - (h^{t})^{2}) \odot \left((\G_{h^{t}} L) \otimes x^{t}\right)
\end{gather}

Some points:
\begin{itemize}
\item The terms \(1 - (h^{t})^{2}\) which occur in \ref{hidden}, \ref{hidden-full} are due to the \(\tanh\) non-linearity. The derivative of the \(\tanh\) function is \(1 - \tanh^{2}(z)\).
\end{itemize}

Now the weight update rule is simple:
\begin{gather}
\displaystyle W_{ho} := W_{ho} - \eta \G_{W_{ho}}L = W_{ho} - \eta \sum_{t=T}^{t=k} (\G_{o^{t}} L) \otimes h^{t} \\
\displaystyle W_{hh} := W_{ho} - \eta \G_{W_{hh}}L = W_{hh} - \eta \sum_{t=T}^{t=k} (1 - (h^{t})^{2}) \odot \left((\G_{h^{t}} L) \otimes (h^{t-1})^{T}\right) \\
\displaystyle W_{xh} := W_{xh} - \eta \G_{W_{xh}}L = W_{xh} - \eta \sum_{t=T}^{t=k} (1 - (h^{t})^{2}) \odot \left((\G_{h^{t}} L) \otimes x^{t}\right)
\end{gather}
\end{itemize}
\end{flushleft}

\subsection*{Part b}
\begin{flushleft}
\begin{itemize}
\item [\textbf{i.}] The network is trained using standard gradient descent, with a weight decay. Additionally, there is a provision to stop back-propagating beyond a certain to avoid vanishing gradients. To prevent exploding gradients, the gradients are clipped to have values between -2 and 2 only. The weight decay parameter is chosen to be \(10^{-4}\) and the learning rate is \(3\times 10^{-4}\). Below is the graph of training loss vs. number of epochs for 25 epochs.

% Insert graph.
\item [\textbf{ii.}]
\item [\textbf{iii.}]
\end{itemize}
\end{flushleft}

\end{document}
