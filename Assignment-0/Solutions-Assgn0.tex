\documentclass{article}
\usepackage[margin=1in]{geometry}
\usepackage{amsmath}
\usepackage{amssymb}
\usepackage{hyperref}
\newcommand{\argmin}{\mathop{\mathrm{argmin}}}
\newcommand{\argmax}{\mathop{\mathrm{argmax}}}
\newcommand{\dO}{\partial}

\title{Solutions to the Assignment - 0 : CS5480 - \\
Deep Learning}
\author{Vishwak Srinivasan\\
\texttt{CS15BTECH11043}}
\date{}

\begin{document}
\maketitle

\section*{Question 1}
\begin{flushleft}
Consider the payout to be a function of \(x\), given by:
\begin{equation}
f(x) = \begin{cases} +100, & \text{if } x = 1 \\ -25, & \text{if } x \neq 1 \end{cases}
\end{equation}

Now the expected payout from the die roll can be given by \(\mathbb{E}[f(X)]\). Since this is a discrete distribution with \(p(x) = \frac{1}{6} \hspace{2mm} \forall x \in \{1, 2, 3, 4, 5, 6\}\):
\begin{equation}
\displaystyle \mathbb{E}[f(X)] = \sum_{x=1}^{6} p(x)f(x) = \frac{100 - 25 - 25 - 25 - 25 - 25}{6} = \frac{-25}{6}
\end{equation}

The expected payout to a visitor is \(\boxed{\text{Rs.} -\frac{-25}{6}}\), which is a loss. Hence, this might not be a good plan, and the organizers will not make money (in expectation).
\end{flushleft}

\section*{Question 2}
\begin{flushleft}
We know that \(f = \langle W^{3}, h^{2} \rangle = \displaystyle \sum_{k=1}^{2} W^{3}_{k}h^{2}_{k}\).
More information:
\begin{itemize}
\item \(\displaystyle h^{2}_{k} = \sigma\left(\sum_{j=1}^{2} W^{2}_{g, k} h^{1}_{g}\right) \)
\item \(\displaystyle h^{1}_{g} = \sigma\left(\sum_{l=1}^{2} W^{1}_{l, g} x_{l}\right) \)
\end{itemize}
Now using chain rule:
\begin{gather}
\label{eq1}
\displaystyle \frac{\dO f}{\dO W^{1}_{i, j}} = \sum_{k=1}^{2}W^{3}_{k}\frac{\dO h^{2}_{k}}{\dO W^{1}_{i, j}} \\
\label{eq2}
\displaystyle \frac{\dO h^{2}_{k}}{\dO W^{1}_{i, j}} = \sigma'\left(\sum_{g=1}^{2} W^{2}_{g,k}h^{1}_{g}\right) \times \left(\frac{\dO h^{1}_{j}}{\dO W^{1}_{i, j}}\right) \\
\label{eq3}
\displaystyle \frac{\dO h^{1}_{j}}{\dO W^{1}_{i, j}} = \sigma'\left(\sum_{l=1}^{2} W^{1}_{l,g}x_{l}\right) \times x_{i}
\end{gather}

Assembling the above equations \ref{eq1}, \ref{eq2} and \ref{eq3}, we get:
\begin{equation}
\displaystyle \frac{\dO f}{\dO W^{1}_{i,j}} = \sum_{k=1}^{2}W^{3}_{k}\left(\sigma'\left(\sum_{g=1}^{2} W^{2}_{g,k}h^{1}_{g}\right) \times \left(W^{2}_{j, k} \left(\sigma'\left(\sum_{l=1}^{2} W^{1}_{l,j}x_{l}\right) \times x_{i}\right) \right) \right)
\end{equation}

If \(\sigma(.)\) is the sigmoid activation:
\begin{equation}
\displaystyle \frac{\dO f}{\dO W^{1}_{i, j}} = \sum_{k=1}^{2}W^{3}_{k} \left(h^{2}_{k} (1 - h^{2}_{k}) \times \left(W^{2}_{j,k} \left(h^{1}_{j} (1 - h^{1}_{j}) \times x_{i}\right) \right)\right)
\end{equation}
\end{flushleft}

\section*{Question 3}
\begin{flushleft}
Note that since we have to compute all pairs of \((i, j)\), this will be an outer product.
\begin{equation}
\Delta^{(2)} := \Delta^{(2)} + \delta^{(3)} a^{(2)^{T}}
\end{equation}
where \(\delta^{(3)} = \nabla_{W^{3}} f\) and \(a^{(2)}\) is the raw output from the first layer (without activations).
\end{flushleft}

\section*{Question 4}
\begin{flushleft}
The first perceptron (bias-less) would learn a function \(f\) by learning the parameters \(w_1, w_2\) to produce a classifier whose boundary is \(w_1 x_1 + w_2 x_2 = 0\). Now, if there was a fixed bias \(( = 1)\), then the resulting classifier with a decision boundary given by \(w_1' x_1 + w_2' x_2 + 1 = 0\). Note that because of the constant 1, for any arbitrary data, there will be a difference in the decision boundary of the bias-less and biased perceptron, thus leading to the same \(f\) not being learned.
\end{flushleft}

\section*{Question 5}
The function in consideration is \(f(\mathbf{x}) = \displaystyle \log\left(\sum_{i=1}^{n} e^{x_i}\right)\)
\subsection*{Part a}
\begin{flushleft}
\begin{equation}
\frac{\dO f}{\dO x_{i}} = \frac{e^{x_i}}{\displaystyle \sum_{i=1}^{n} e^{x_i}} \quad \forall i \in 1, \ldots, n
\end{equation}
\end{flushleft}

\subsection*{Part b}
\begin{flushleft}
\begin{equation}
\nabla f(\mathbf{x}) = \langle \ldots, \frac{\dO f}{\dO x_{i}}, \ldots\rangle = \left(\frac{1}{\displaystyle \sum_{i=1}^{n} e^{x_i}}\right) \langle e^{x_1}, e^{x_2}, \ldots, e^{x_n}\rangle
\end{equation}
\end{flushleft}

\subsection*{Part c}
\begin{flushleft}
If \(f : \mathbb{R} \to \mathbb{R}\), \(g : \mathbb{R}^{n} \to \mathbb{R}\) and \(h : \mathbb{R}^{n} \to \mathbb{R}\) are functions such that \(h = f \circ g\), then:
\begin{equation}
\nabla h(\mathbf{x}) = \nabla (f \circ g)(\mathbf{x}) = \frac{d(f \circ g)}{dg(\mathbf{x})} \nabla g(\mathbf{x})
\end{equation}
\end{flushleft}

\section*{Question 6}
\begin{flushleft}
Consider a univariate regression task with a mean-squared loss function for the neural network. Mathematically, this can be formulated as:
\begin{equation}
\displaystyle \text{Loss}(\mathcal{D} = ((X_1, y_1), (X_2, y_2), \ldots, (X_m, y_m)) = \frac{1}{m}\sum_{i=1}^{m} (y_{i} - WX_{i})^{2}
\end{equation}
Here \(W\) is the product of all the weights in the neural network as \(W = W_{H}W_{H-1}\ldots W_{1}\).

We seek to minimize this loss function by using gradient descent over all the parameters of the network. This can be written as follows:
\begin{equation}
\label{gd}
W^{*}_{H}, W^{*}_{H-1}, \ldots W^{*}_{1} = \argmin_{W_{H}, W_{H-1}, \ldots, W_{1}} \frac{1}{m}\sum_{i=1}^{m} (y_{i} - WX_{i})^{2}
\end{equation}

Now, for the same univariate regression problem, assume that the differences between \(y_{i} - WX_{i} = \epsilon_{i}\) be modelled as \textit{i.i.d.} zero-mean Gaussian random variable \(\mathcal{N}(0, \sigma^{2})\). The likelihood of the function given the \(\epsilon\)'s can be given by:
\begin{equation}
\displaystyle L = \prod_{i=1}^{m} \frac{1}{\sqrt{2\pi} \sigma} e^{-\left(\frac{\epsilon_{i}^{2}}{2\sigma^{2}}\right)}
\end{equation}

Now, the same likelihood computed for a given \(W\) is:
\begin{equation}
\displaystyle L(y | X, W) = \prod_{i=1}^{m} \frac{1}{\sqrt{2\pi} \sigma} e^{-\left(\frac{(y_{i} - WX_{i})^2}{2\sigma^{2}}\right)}
\end{equation}

We would like to find the best set of parameters \(W_{H}, W_{H-1}, \ldots, W_{1}\), to maximize this likehilood (or the log-likelihood)
\begin{multline}
\label{mle}
\displaystyle W^{*}_{H}, W^{*}_{H-1}, \ldots W^{*}_{1} = \argmax_{W_{H}, W_{H-1}, \ldots, W_{1}} \log L(y | X, W) = \\
\argmax_{W_{H}, W_{H-1}, \ldots, W_{1}} -\sum_{i=1}^{m} \log(\sqrt{2\pi}\sigma) - \frac{1}{2\sigma^{2}} \sum_{i=1}^{m} (y_{i} - WX_{i})^2 = \argmin_{W_{H}, W_{H-1}, \ldots, W_{1}} \sum_{i=1}^{m} (y_{i} - WX_{i})^2
\end{multline}

Note that the equations \ref{gd} and \ref{mle} produce the same solutions, since the functions are the same (except for the scaling in \ref{gd} due to the mean). Hence the equivalence between backpropagation (or gradient descent) and maximum likelihood estimation has been shown when:
\begin{itemize}
\item The deviation between the predictions and targets are \textit{i.i.d.} zero-centered Gaussian random variables
\item The (mean-)squared loss function is taken into account.
\end{itemize}
\end{flushleft}

\section*{Question 7}
A perceptron is a harder version of a logistic regression classifier, not permitting any slack. Both these classifiers learn a standard linear boundary separating (not in all cases) two classes of points. However:
\begin{itemize}
\item The perceptron has a more basic structure of learning based on Widrow-Hoff's learning rule, wherein the weights are altered by the amount of "misclassification" (which is hard here (0-1 loss) based). On the other hand, the logistic regression classifier allows for "softer" training, by penalizing the probabilities of classifying a wrong class based on a different loss function (binary cross entropy).
\item I feel that the inference of both these classifiers would be very similar despite the solutions not being so, because for a perceptron if \(w^{T} x > 0\), then \(\alpha w^{T} x > 0, \forall \alpha \in \mathbb{R}^{++}\).
\end{itemize}

\section*{Question 8}
All these times include time taken for type-setting as well.
\begin{itemize}
\item Time spent on question 1: 5 minutes
\item Time spent on question 2: 20 minutes
\item Time spent on question 3: 10 minutes
\item Time spent on question 4: 20 minutes
\item Time spent on question 5: 15 minutes
\item Time spent on question 6: 10 minutes
\end{itemize}
\end{document}
